%%%%%%%%%%%%%%%%%%%%%%%%%%%%%%%%%%%%%%%%%
% University Assignment Title Page 
% LaTeX Template
% Version 1.0 (27/12/12)
%
% This template has been downloaded from:
% http://www.LaTeXTemplates.com
%
% Original author:
% WikiBooks (http://en.wikibooks.org/wiki/LaTeX/Title_Creation)
%
% License:
% CC BY-NC-SA 3.0 (http://creativecommons.org/licenses/by-nc-sa/3.0/)
% 
% Instructions for using this template:
% This title page is capable of being compiled as is. This is not useful for 
% including it in another document. To do this, you have two options: 
%
% 1) Copy/paste everything between \begin{document} and \end{document} 
% starting at \begin{} and paste this into another LaTeX file where you 
% want your title page.
% OR
% 2) Remove everything outside the \begin{titlepage} and \end{titlepage} and 
% move this file to the same directory as the LaTeX file you wish to add it to. 
% Then add %%%%%%%%%%%%%%%%%%%%%%%%%%%%%%%%%%%%%%%%%
% University Assignment Title Page 
% LaTeX Template
% Version 1.0 (27/12/12)
%
% This template has been downloaded from:
% http://www.LaTeXTemplates.com
%
% Original author:
% WikiBooks (http://en.wikibooks.org/wiki/LaTeX/Title_Creation)
%
% License:
% CC BY-NC-SA 3.0 (http://creativecommons.org/licenses/by-nc-sa/3.0/)
% 
% Instructions for using this template:
% This title page is capable of being compiled as is. This is not useful for 
% including it in another document. To do this, you have two options: 
%
% 1) Copy/paste everything between \begin{document} and \end{document} 
% starting at \begin{} and paste this into another LaTeX file where you 
% want your title page.
% OR
% 2) Remove everything outside the \begin{titlepage} and \end{titlepage} and 
% move this file to the same directory as the LaTeX file you wish to add it to. 
% Then add %%%%%%%%%%%%%%%%%%%%%%%%%%%%%%%%%%%%%%%%%
% University Assignment Title Page 
% LaTeX Template
% Version 1.0 (27/12/12)
%
% This template has been downloaded from:
% http://www.LaTeXTemplates.com
%
% Original author:
% WikiBooks (http://en.wikibooks.org/wiki/LaTeX/Title_Creation)
%
% License:
% CC BY-NC-SA 3.0 (http://creativecommons.org/licenses/by-nc-sa/3.0/)
% 
% Instructions for using this template:
% This title page is capable of being compiled as is. This is not useful for 
% including it in another document. To do this, you have two options: 
%
% 1) Copy/paste everything between \begin{document} and \end{document} 
% starting at \begin{} and paste this into another LaTeX file where you 
% want your title page.
% OR
% 2) Remove everything outside the \begin{titlepage} and \end{titlepage} and 
% move this file to the same directory as the LaTeX file you wish to add it to. 
% Then add %%%%%%%%%%%%%%%%%%%%%%%%%%%%%%%%%%%%%%%%%
% University Assignment Title Page 
% LaTeX Template
% Version 1.0 (27/12/12)
%
% This template has been downloaded from:
% http://www.LaTeXTemplates.com
%
% Original author:
% WikiBooks (http://en.wikibooks.org/wiki/LaTeX/Title_Creation)
%
% License:
% CC BY-NC-SA 3.0 (http://creativecommons.org/licenses/by-nc-sa/3.0/)
% 
% Instructions for using this template:
% This title page is capable of being compiled as is. This is not useful for 
% including it in another document. To do this, you have two options: 
%
% 1) Copy/paste everything between \begin{document} and \end{document} 
% starting at \begin{} and paste this into another LaTeX file where you 
% want your title page.
% OR
% 2) Remove everything outside the \begin{titlepage} and \end{titlepage} and 
% move this file to the same directory as the LaTeX file you wish to add it to. 
% Then add \input{./title_page_1.tex} to your LaTeX file where you want your
% title page.
%
%%%%%%%%%%%%%%%%%%%%%%%%%%%%%%%%%%%%%%%%%

%----------------------------------------------------------------------------------------
%	PACKAGES AND OTHER DOCUMENT CONFIGURATIONS
%----------------------------------------------------------------------------------------

%\documentclass[12pt]{article}
%
%\begin{document}

\begin{titlepage}

\newcommand{\HRule}{\rule{\linewidth}{0.5mm}} % Defines a new command for the horizontal lines, change thickness here

\center % Center everything on the page
 
%----------------------------------------------------------------------------------------
%	HEADING SECTIONS
%----------------------------------------------------------------------------------------

\textsc{\LARGE Department of Financial Mathematics}\\[1.5cm] % Name of your university/college
\textsc{\Large }\\[0.5cm] % Major heading such as course name
\textsc{\large }\\[0.5cm] % Minor heading such as course title

%----------------------------------------------------------------------------------------
%	TITLE SECTION
%----------------------------------------------------------------------------------------

\HRule \\[0.5cm]
{ \huge \bfseries Compare the quality of forecasting models for value at risk}\\[0.1cm] % Title of your document
\HRule \\[1.5cm]
\vspace{2.5cm}
 
%----------------------------------------------------------------------------------------
%	AUTHOR SECTION
%----------------------------------------------------------------------------------------


\begin{minipage}{0.4\textwidth}
\begin{flushleft} \large
\emph{Author:}\\
Hafees Adebayo \textsc{Yusuff} % Your name
\end{flushleft}
\end{minipage}
~
\begin{minipage}{0.4\textwidth}
\begin{flushleft} \large
\emph{Supervisor:} \\
Prof. Ralf \textsc{Korn} % Supervisor's Name
\end{flushleft}
\end{minipage}\\[4cm]

% If you don't want a supervisor, uncomment the two lines below and remove the section above
%\Large \emph{Author:}\\
%John \textsc{Smith}\\[3cm] % Your name

%----------------------------------------------------------------------------------------

%	DATE SECTION
%----------------------------------------------------------------------------------------

{\large \today}\\[3cm] % Date, change the \today to a set date if you want to be precise
\vspace{4.5cm}

%----------------------------------------------------------------------------------------
%	LOGO SECTION
%----------------------------------------------------------------------------------------
\begin{minipage}[b]{0.4\textwidth}
%\begin{flushleft} \large
\begin{center}
\includegraphics[width=0.5\textwidth]{figures/download}\\[1cm] % Include a department/university logo - this will require the graphicx package
%\end{flushleft}
\end{center}
\end{minipage}
%~
%\begin{minipage}[b]{0.4\textwidth}
%\begin{flushleft} \large
%\includegraphics[width=0.5\textwidth]{figures/download (1)}
%\end{flushleft}
%\end{minipage}

%----------------------------------------------------------------------------------------

\vfill % Fill the rest of the page with whitespace

\end{titlepage}

\begin{titlepage}
	This thesis is written as a requirement for the completion of my degree of Master of Science at Technical University of Kaiserslautern.
\end{titlepage}
 to your LaTeX file where you want your
% title page.
%
%%%%%%%%%%%%%%%%%%%%%%%%%%%%%%%%%%%%%%%%%

%----------------------------------------------------------------------------------------
%	PACKAGES AND OTHER DOCUMENT CONFIGURATIONS
%----------------------------------------------------------------------------------------

%\documentclass[12pt]{article}
%
%\begin{document}

\begin{titlepage}

\newcommand{\HRule}{\rule{\linewidth}{0.5mm}} % Defines a new command for the horizontal lines, change thickness here

\center % Center everything on the page
 
%----------------------------------------------------------------------------------------
%	HEADING SECTIONS
%----------------------------------------------------------------------------------------

\textsc{\LARGE Department of Financial Mathematics}\\[1.5cm] % Name of your university/college
\textsc{\Large }\\[0.5cm] % Major heading such as course name
\textsc{\large }\\[0.5cm] % Minor heading such as course title

%----------------------------------------------------------------------------------------
%	TITLE SECTION
%----------------------------------------------------------------------------------------

\HRule \\[0.5cm]
{ \huge \bfseries Compare the quality of forecasting models for value at risk}\\[0.1cm] % Title of your document
\HRule \\[1.5cm]
\vspace{2.5cm}
 
%----------------------------------------------------------------------------------------
%	AUTHOR SECTION
%----------------------------------------------------------------------------------------


\begin{minipage}{0.4\textwidth}
\begin{flushleft} \large
\emph{Author:}\\
Hafees Adebayo \textsc{Yusuff} % Your name
\end{flushleft}
\end{minipage}
~
\begin{minipage}{0.4\textwidth}
\begin{flushleft} \large
\emph{Supervisor:} \\
Prof. Ralf \textsc{Korn} % Supervisor's Name
\end{flushleft}
\end{minipage}\\[4cm]

% If you don't want a supervisor, uncomment the two lines below and remove the section above
%\Large \emph{Author:}\\
%John \textsc{Smith}\\[3cm] % Your name

%----------------------------------------------------------------------------------------

%	DATE SECTION
%----------------------------------------------------------------------------------------

{\large \today}\\[3cm] % Date, change the \today to a set date if you want to be precise
\vspace{4.5cm}

%----------------------------------------------------------------------------------------
%	LOGO SECTION
%----------------------------------------------------------------------------------------
\begin{minipage}[b]{0.4\textwidth}
%\begin{flushleft} \large
\begin{center}
\includegraphics[width=0.5\textwidth]{figures/download}\\[1cm] % Include a department/university logo - this will require the graphicx package
%\end{flushleft}
\end{center}
\end{minipage}
%~
%\begin{minipage}[b]{0.4\textwidth}
%\begin{flushleft} \large
%\includegraphics[width=0.5\textwidth]{figures/download (1)}
%\end{flushleft}
%\end{minipage}

%----------------------------------------------------------------------------------------

\vfill % Fill the rest of the page with whitespace

\end{titlepage}

\begin{titlepage}
	This thesis is written as a requirement for the completion of my degree of Master of Science at Technical University of Kaiserslautern.
\end{titlepage}
 to your LaTeX file where you want your
% title page.
%
%%%%%%%%%%%%%%%%%%%%%%%%%%%%%%%%%%%%%%%%%

%----------------------------------------------------------------------------------------
%	PACKAGES AND OTHER DOCUMENT CONFIGURATIONS
%----------------------------------------------------------------------------------------

%\documentclass[12pt]{article}
%
%\begin{document}

\begin{titlepage}

\newcommand{\HRule}{\rule{\linewidth}{0.5mm}} % Defines a new command for the horizontal lines, change thickness here

\center % Center everything on the page
 
%----------------------------------------------------------------------------------------
%	HEADING SECTIONS
%----------------------------------------------------------------------------------------

\textsc{\LARGE Department of Financial Mathematics}\\[1.5cm] % Name of your university/college
\textsc{\Large }\\[0.5cm] % Major heading such as course name
\textsc{\large }\\[0.5cm] % Minor heading such as course title

%----------------------------------------------------------------------------------------
%	TITLE SECTION
%----------------------------------------------------------------------------------------

\HRule \\[0.5cm]
{ \huge \bfseries Compare the quality of forecasting models for value at risk}\\[0.1cm] % Title of your document
\HRule \\[1.5cm]
\vspace{2.5cm}
 
%----------------------------------------------------------------------------------------
%	AUTHOR SECTION
%----------------------------------------------------------------------------------------


\begin{minipage}{0.4\textwidth}
\begin{flushleft} \large
\emph{Author:}\\
Hafees Adebayo \textsc{Yusuff} % Your name
\end{flushleft}
\end{minipage}
~
\begin{minipage}{0.4\textwidth}
\begin{flushleft} \large
\emph{Supervisor:} \\
Prof. Ralf \textsc{Korn} % Supervisor's Name
\end{flushleft}
\end{minipage}\\[4cm]

% If you don't want a supervisor, uncomment the two lines below and remove the section above
%\Large \emph{Author:}\\
%John \textsc{Smith}\\[3cm] % Your name

%----------------------------------------------------------------------------------------

%	DATE SECTION
%----------------------------------------------------------------------------------------

{\large \today}\\[3cm] % Date, change the \today to a set date if you want to be precise
\vspace{4.5cm}

%----------------------------------------------------------------------------------------
%	LOGO SECTION
%----------------------------------------------------------------------------------------
\begin{minipage}[b]{0.4\textwidth}
%\begin{flushleft} \large
\begin{center}
\includegraphics[width=0.5\textwidth]{figures/download}\\[1cm] % Include a department/university logo - this will require the graphicx package
%\end{flushleft}
\end{center}
\end{minipage}
%~
%\begin{minipage}[b]{0.4\textwidth}
%\begin{flushleft} \large
%\includegraphics[width=0.5\textwidth]{figures/download (1)}
%\end{flushleft}
%\end{minipage}

%----------------------------------------------------------------------------------------

\vfill % Fill the rest of the page with whitespace

\end{titlepage}

\begin{titlepage}
	This thesis is written as a requirement for the completion of my degree of Master of Science at Technical University of Kaiserslautern.
\end{titlepage}
 to your LaTeX file where you want your
% title page.
%
%%%%%%%%%%%%%%%%%%%%%%%%%%%%%%%%%%%%%%%%%

%----------------------------------------------------------------------------------------
%	PACKAGES AND OTHER DOCUMENT CONFIGURATIONS
%----------------------------------------------------------------------------------------

%\documentclass[12pt]{article}
%
%\begin{document}

\begin{titlepage}
	

\newcommand{\HRule}{\rule{\linewidth}{0.5mm}} % Defines a new command for the horizontal lines, change thickness here

\center % Center everything on the page
 
%----------------------------------------------------------------------------------------
%	HEADING SECTIONS
%----------------------------------------------------------------------------------------
\begin{minipage}[b]{0.4\textwidth}
	%\begin{flushleft} \large
	\begin{center}
		\includegraphics[width=0.5\textwidth]{figures/download}\\[1cm] % Include a department/university logo - this will require the graphicx package
		%\end{flushleft}
	\end{center}
\end{minipage}
\vspace{1.5cm}



\textsc{\LARGE Department of Financial Mathematics}\\[1.5cm] % Name of your university/college
\textsc{\Large }\\[0.5cm] % Major heading such as course name
\textsc{\large }\\[0.5cm] % Minor heading such as course title

%----------------------------------------------------------------------------------------
%	TITLE SECTION
%----------------------------------------------------------------------------------------

\HRule \\[0.5cm]
{ \huge \bfseries Comparison of Forecasting Models for Value at Risk}\\[0.1cm] % Title of your document
\HRule \\[1.5cm]
\vspace{2.5cm}
 
%----------------------------------------------------------------------------------------
%	AUTHOR SECTION
%----------------------------------------------------------------------------------------


\begin{minipage}{0.4\textwidth}
\begin{flushleft} \large
\emph{Author:}\\
Hafees Adebayo \textsc{Yusuff} % Your name
\end{flushleft}
\end{minipage}
~
\begin{minipage}{0.4\textwidth}
\begin{flushleft} \large
\emph{Supervisor:} \\
Prof. Ralf \textsc{Korn} % Supervisor's Name
\end{flushleft}
\end{minipage}\\[4cm]

% If you don't want a supervisor, uncomment the two lines below and remove the section above
%\Large \emph{Author:}\\
%John \textsc{Smith}\\[3cm] % Your name

%----------------------------------------------------------------------------------------

%	DATE SECTION
%----------------------------------------------------------------------------------------

{\large \today}\\[3cm] % Date, change the \today to a set date if you want to be precise
\vspace{2.5cm}


A thesis submitted in fulfilment of the requirements for the degree of
Master of Science

%----------------------------------------------------------------------------------------
%	LOGO SECTION
%----------------------------------------------------------------------------------------

%~
%\begin{minipage}[b]{0.4\textwidth}
%\begin{flushleft} \large
%\includegraphics[width=0.5\textwidth]{figures/download (1)}
%\end{flushleft}
%\end{minipage}

%----------------------------------------------------------------------------------------

\vfill % Fill the rest of the page with whitespace
\end{titlepage}



\thispagestyle{empty}

\newpage\null\thispagestyle{empty}\newpage


\begin{titlepage}
	\textbf{\LARGE Declaration of Authorship}\newline\newline
	

		
		I, Hafees Adebayo YUSUFF, hereby declare the following thesis titled "Comparison of forecasting models for value at risk" to be my own work and I confirm that:
			\begin{itemize}
		\item[$\bullet$] The thesis I am submitting is entirely my own work except where otherwise indicated.
		
		\item[$\bullet$] It has not been submitted, either partially or in full, for a qualification at this or any other University.
	
	\item[$\bullet$] I have clearly signalled the presence of all material I have quoted from other sources,
	including any diagrams, charts, tables or graphs.
	
	\item[$\bullet$] I have acknowledged appropriately any assistance I have received.
	
	\end{itemize}
	
	\vspace*{4em}\noindent
	\hfill%
	\begin{tabular}[t]{c}
		\rule{10em}{0.4pt}\\ Signature
	\end{tabular}%
	\hfill%
	\begin{tabular}[t]{c}
		\rule{10em}{0.4pt}\\ Date
	\end{tabular}%
	\hfill\strut


\end{titlepage}

\thispagestyle{empty}

\newpage\null\thispagestyle{empty}\newpage

\begin{titlepage}
	\textbf{\LARGE Abstract}\newline\newline
Financial institutions need to have enough capacity to meet their responsibilities and sop up unexpected losses. Since they are exposed to risks, managing them is very important. Several methods have been provided for managing risk, of which Value at Risk (VaR) is the most common for market risk. VaR is a statistical method used to measure the amount of potential loss that could happen in an investment portfolio over a specified period of time under normal market conditions.\\\\ This study compares some VaR estimation methods: Historical simulation, Garch (1,1) model and Long short term memory (LSTM) neural network using the Japan, UK and US stock markets. Each stock market contains 8476 daily log-returns from 05/01/1988 to 30/06/2020. For the Historical simulation and Garch (1,1) model, we use the first 7090 days as our rolling window. As for the LSTM VaR model, the first 7000 daily log-returns (83\% of data) of each series are used for training, 1400 (20\%of data for training) are used for validation while the remaining 1386 are used for testing. From the kupiec test, Historical simulation outperforms other models as it is accepted for both the 95\% and 99\% confidence level. Only the Garch (1,1) model with 95\% confidence level is accepted for all considered stock markets, while that of 99\% is rejected. The LSTM VaR model with 95\% confidence interval is accepted for S\&P 500 (US) and  FTSE 100 (UK), but rejected for NIKKEI 225 (Japan). However, the Kupiec test disapproves the LSTM VaR model with 99\% confidence level for all the three stocks markets.


\end{titlepage}
\thispagestyle{empty}

\newpage\null\thispagestyle{empty}\newpage

